In this thesis a heuristical approach to solve the scheduling of vehicles in a cyclic flexible manufacturing system was considered. The use of an iterated local search returned promising results in decent enough time staying just under two minutes for the benchmark version of the heuristic. However, due to the fact that the problem considered in this thesis has - to the best of the author's knowledge - never been studied before in literature, any remarks based on the quality of the achieved values have no real point of comparison. The values obtained in this thesis may therefore serve as a point of orientation for the performance of future algorithms for obtaining optimal and heuristical solutions. One of the big problems the results of this thesis will have to face is most definitely if the time saved is actually valuable enough to warrant an increase in the objective value of one percent. Since the underlying problem is based on the scheduling of AGVs in for example a large factory, a significant reduction in time might not be as valuable as a one percent increase in total cost, since scheduling plans are generally set in advance and have the advantage of actually having some leeway concerning the speed at which solutions are required, while even a single percent of change in cost might mean a difference in millions. As previously stated in this paragraph, however, the test case is still very small compared to some of the real world problems that big companies planning the introduction of totally automated transport might face.