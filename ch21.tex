The category of flexible flow shop scheduling problems has a long history in the literature and has been addressed in a great amount of papers. Arthanari and Ramamurthy are credited to be among the first to have brought up the problem in their paper "An extension of the two-machine sequencing problem" \cite{arth1971}. However there were even earlier attempts at similar problems, notably in 1954 by Dantzig and Fulkerson \cite{dantzig1954} in which they had a look at the problem of minimizing the amount of carriers to meet a fixed schedule. In the paper they showed that finding the minimum amount of tankers needed is solveable as a linear problem.

Since then a lot of different angles of the initial problem have been looked at, an article published by Wang \cite{wang2005} in 2005 provides a general overview of different approaches and special cases. More recently Jungwattanakit et al \cite{jung2009} published a comparison of scheduling algorithms for flexible flow shop problems specifically looking at problems with unrelated parallel machines, setup times and dual criteria.

The biggest subcategory of flexible flow shop problems that the problem in this thesis falls into is the category of cyclic flexible flow shop. Among the first to pick up this specific version of the flexible flow shop scheduling problems were Blazewicz et al \cite{blazewicz1994} in 1994. In an unpublished paper Blazewicz and Pawlak also constructed the specific variation of the cyclic flexible flow shop that poses the foundation upon which this thesis is built \cite{blazewicz1111} and proposed a method to solve the problem using a circular arc-graph model.\cite{blazewicz198}

Since then most notably the works of Kats and Levner added knowledge surrounding the problem. Kats and Levner wrote four papers together that focused on very similar problems to the given one. Kats' paper "Minimizing the number of robots to meet a given cyclic schedule" \cite{kats1997b} shares the idea of assigning transportation vehicles to a given schedule, but the problem differs in two main points: Instead of the transportation vehicles having a no-wait clause, the jobs are not allowed to wait at the station, there is no outgoing queue allowed. The second major difference is that although the robots have to stick to a cycle of sequence of moves, that sequence can be generated from all available moves. Therefore instead of all robots following the same cycle, each robot follows its own. This version of the cyclic flow shop is also the basis for Kats' second paper published in the same year \cite{kats1997a} in which the researchers introduce a strongly polynomial algorithm to solve the problem.

Also in 1997 Levner, Kats and Levit proposed an improved algorithm for cyclic flow shop scheduling in a robotic cell \cite{levner1997}. The main objective in the paper is to find an optimal cycle for just a single robot. This poses an alternate approach instead of simply cycling the machines in the order that they are visited in the flow shop. The approach described in this paper provides a theoretical bridge between the problem focused on in this thesis with multiple vehicles on a set course and the problem described by Kats and Levner \cite{kats1997b} with multiple vehicles each of which to be sent on an independant one.

In 2010 Levner et al published a survey about the complexity of cyclic scheduling problems in which they further go into detail about the progress that has been made on the field of cyclic scheduling as a whole. They distinguish cyclic scheduling problems into three main categories: cyclic jobshop, cyclic flowshop and cyclic project scheduling problems. The problem looked into in this thesis falls into the second category which according to Levner et al is NP-hard in general, however there are specific cases that are solveable in polynomial time \cite{levner2010}.