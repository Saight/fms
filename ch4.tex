In this section the paper will go into depth as to how the problem was approached from a mathematical perspective. To be able to approach the problem
with a programmed algorithm, some definitions have to be made beforehand. As described closer in previous chapters, the general idea is that AGVs
are released onto a track, where they will run in a neverending circle. Upon reaching a machine or the depot the currently loaded job is unloaded and
the job that is scheduled to be processed the earliest on the next machine is selected from the jobs waiting in the outgoing queue at the machine.
Jobs unloaded at a machine are placed in the incoming queue. Note that this queue does not have a real order as the schedule of jobs on the machines is
given, meaning that even though a job might have been the first to arrive at a machine, it might not get processed at said machine until after some
other jobs have been. The objective of the program is to minimize the maximum completion time for a given amount of AGVs \(V\). The objective function can therefore be
depicted as:

\begin{equation}
\mathrm{min}\>C_{max}
\end{equation}

For this formulation a relatively long list of constraints has to be set, which will now be explained in detail. Constraint (1) is set to make sure
that every available vehicle has to start before a full lap could have been made, since if a vehicle was to start at a later point in time it would
simply miss a full lap that could have been achieved by letting the vehicle start roundtime \(R\) units of time earlier.

\(s.t.:\)
\begin{equation}
\sum_{t=0}^{R-1}s_{v t} = 1\ldots \forall v.
\end{equation}
\begin{equation}
s_{v t}\in\{0,1\}
\end{equation}

In addition to setting the vehicle starting time to be in the range of the optimal starting time, this restriction also guarantees that each vehicle
will start at only one point in time which has to hold true to be able to determine the position of the vehicle throughout the rest of the program.