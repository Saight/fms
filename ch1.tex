%Formulation of the problem

The problem of a cylcic flexible manufactoring system has certain limitations, which will be expanded and described in this chapter. The baseline
of the overall problem is that vehicles that constantly go in a cycle, are used to carry jobs from one machine to the next. Extensive reasearch 
has been made in various directions considering flexible manufactoring systems. Many of these papers focus on the scheduling the tasks of each of
the jobs on the machines (e.g. PAPER,PAPER,PAPER)
This paper will only consider
a flow shop approach in which the machines are already sorted in the order, in which the jobs need to be processed. Due to the cyclic nature of
this special case of the problem, the AGVs are routed to drive on a steady course and because of start and depot being the same point, the AGVs
are constantly repeating laps of said track. This in turn enables us to know the position of an AGV at any given point in the system solely by
setting a starting time at which the AGV is "released". The actual position in the system is then acquired by calculating the overall time minus
the starting time modulo the lap time.

pos(t) = (Tm + time) percent laptime

Since the schedule of the jobs on the machines themselves is considered to be given, after setting the AGVs to automatically pick up the job 
which is next in the schedule on the next machine and waiting in the outgoing queue of the current machine, the only input needed to be able
to achieve a solution is a starting time for each of the AGVs.

Before it is possible to formulate an approach to solve this problem, some rules have to be set first, especially considering the handling of
jobs in outgoing queues (jobs which have been completed at a machine but have not yet been picked up by an AGV). Since the system is set to be
in constant movement, a machine cannot simply wait for a job no yet finished. Therefore a rule has to be set as to which job is to be picked out
of those waiting in line. This rule requires the machine to pick up the next job in chronological order that is both scheduled to follow the job
currently being worked on at the next stage, and waiting in the outgoing queue of the current machine. The following example shows as to how
picking up jobs in a different order might worsen the solution.

Consider a problem with a single AGV and a schedule on machine n+1 set to be [1,2,3]. Jobs 2 and 3 are currently waiting to be picked up 
in the outgoing queue of machine n, while job 1 is still currently being processed on machine n. Considering the lemma an incoming AGV has to pick
up job 2 as it is the next job which is supposed to be processed on machine n+1 and sitting in the outgoing queue of machine n. In this case the 
AGV now picks up job 3 instead and delivers it to machine n+1, where job 3 is now waiting in the incoming queue. While the AGV is continuing its
lap job 1 has now been finished on machine n and is ready for pick up. As the AGV returns to machine n the job 1 is now picked up and delivered
to the machine n+1, where it can immediately start to be processed. Somewhere during the continuation of the lap of the AGV job1 is finished on
machine n+1 and the machine is free to process a new job. Here however it is unable to do so as in the incoming queue there is still only the
job 3, which is supposed to be processed only after the job 2. If the AGV had picked up job 2 in the first round the job could have been started
now, so the decision to load job 3 led to an increase in makespan that is at least the minimum of the processing time of job 2 on the second
machine and the time it takes the AGV to pick up and deliver job 2.

The original problem which laid the foundation for this paper, as described by AUTHOR in PAPER, differs from the problem the actually examined
problem in another small but important feature: In the paper by AUTHOR the goal was to minimize the amount of vehicles it takes to be able to
deliver all jobs in time to satisfy a given schedule with the amount of allowed laps being given. The approach in this thesis is that there is no
given deadline known beforehand. The idea is to offer multiple solutions with differing amounts of AGVs given. Each of these solutions will assign
a makespan to a given amount of vehicles, which is defined by the sum of time units passed from the start of the problem until all jobs have been
completed and have been transported back to the depot. These solutions can then be ranked e.g. via
a cost function that takes vehicles and time needed into consideration or, if the goal is to finish ahead of a given deadline, the cost function
can be set to infinite when exceeding the threshold, so simply the solution with the fewest vehicles that finishes in time will be selected.

As briefly mentioned in the previous paragraph, this paper reformulates the problem posed in the paper by AUTHOR to be multidimensional. This
approach has been chosen because it increases the flexibility of the whole approach. While the approach with a given deadline that must be met
is certainly not only suitable for a rare case as these problems occur very often in all kinds of fields, a two-dimensional approach which also
looks at the total makespan introduces the possibility of a deadline that may be broken for a cost (late surcharge) or even simpler a model in
which the return is based on producing as fast as possible.
