%Vorlage für akademische Arbeiten
%erstellt von Mag. Martin Glatz
%mathematik@oehunigraz.at
%mathematik.oehunigraz.at
%ig-mathe (Studienvertretung Mathematik Uni Graz)
%Stand: März 2015

% *****************************************************************
% % ********************** Präambel *******************************
% % ***************************************************************


%===================================================================
%Dokumentklasse...
%===================================================================
\documentclass[
    fontsize=10pt,    %Standardschriftgröße auf 10 pt gesetzt
    a4paper,      %Papierformat: b=148mm h=210mm
    parskip=half-,    %kein Absatzeinzug, stattdessen Höhe einer halben Zeile.
    twoside,      %zweiseitiges Layout (Seitenzahlen außen)
    headings=big,  %große überschriften samt abstände
    appendixprefix=true,
    ]{scrreprt}    %koma-script: europäische Formatvorlage für  scrreprt (chapter-ebene vorhanden)



%===================================================================
%Standard-Codierungen und Schriften...
%===================================================================

\usepackage[utf8]{inputenc}  %Zeichencodierung in den Textfiles
\usepackage[english,ngerman]{babel}  %deutsche Rechtschreibung, Worttrennung
\usepackage[T1]{fontenc}     %Schriftcodierung
\usepackage{lmodern}      %LaTeX-Schriftpaket laden (lmodern)
%\usepackage{microtype} % für zeilenumbrüche etc, verträgt sich nicht mit  allen schriften. Einfach probieren oder in der dokumentation nachlesen



\input{Vorspann/pakete}



\setcounter{secnumdepth}{4}
\setcounter{tocdepth}{4}

\usepackage[
  colorinlistoftodos,
  textwidth=3.5cm,
  textsize=footnotesize]{todonotes}  %notizen im pdf



%===================================================================
%Einbinden der anderen Präambel-Dateien
%===================================================================



\input{Vorspann/design}
\input{Vorspann/befehle-tabellen}
\input{Vorspann/befehle-mathe}
\input{Vorspann/befehle-umgebungen}

%===================================================================
%Setup für Verlinkungen und Farben etc.
%===================================================================
\usepackage[
  linktocpage=true, 
  colorlinks=true,
  linkcolor=farbe1,
  citecolor=farbe1,
  urlcolor=farbe1
  ]{hyperref}

%===================================================================
%Einstellungen Literatur: Momentan Numerisch. Bei Bedarf ändern
%===================================================================

\usepackage
  [
  backend=biber,  %    sortierprogramm
  %backend=bibtex, %alternatives sortierprogramm. hilfsdaten vor umstellung löschen
  style=numeric-comp,
  maxbibnames=99,
  maxcitenames=1,
  urldate=long,
  sorting=none
  ]
  {biblatex}
\addbibresource{literatur-da.bib} %laden der literaturliste
\DefineBibliographyStrings{ngerman}{andothers={et\addabbrvspace al\adddot}}
\DefineBibliographyStrings{ngerman}{urlseen={abgerufen am}}









% *****************************************************************
% % ********************** Präambel *******************************
% % ***************************************************************

\begin{document}

\pagenumbering{Roman}

\input{beginn/Titelseite} %titelblatt
\clearpage

\input{beginn/Abstract} %abstract englisch
\clearpage

\input{beginn/Kurzfassung} %abstract deutsch
\clearpage


%===================================================================
%Gestaltung Inhaltsverzeichnis
%===================================================================


\setcounter{secnumdepth}{4}  %nummerierte ebenen
\setcounter{tocdepth}{4}  %angeführte ebenen im verzeichnis

\tocsetup{
  title/font = \huge\color{farbe1},
  part/font = \large\bfseries\color{farbe1},
  part/number/width=8mm,
  part/number/after = .,
  chapter/number/after = .,
  chapter/indent=8mm,
  chapter/number/width=8mm,  
  section/number/after = .,
  section/indent = 16mm,
  section/number/width=10mm,
  subsection/indent=26mm,
  subsection/number/after = .,
  subsection/number/width=10mm,
  subsubsection/indent=36mm,
  subsubsection/number/after = .,
  subsubsection/number/width=13mm
  }
\listofsetup{lot}{
  parskip=0cm,
  title/font = \huge\color{farbe1},
  table/number/after = .,
  table/number/width=13mm,  
  }  
\listofsetup{lof}{
  parskip=0cm,
  title/font = \huge\color{farbe1},
  figure/number/after = .,  
  figure/number/width=13mm,}



%Nützliche Befehle fürs toc: 

%\addtocontents{toc}{\protect\clearpage} % fügt Zeilenumbruch im Inhaltsverzeichnis nach dem vorangegangenen Eintrag durch. Muss also im Fließtext bzw. zwischen den Überschriften stehen

%\addtocontents{toc}{\protect\enlargethispage{2\baselineskip}} %fügt nach dem vorigen toc-Eintrag den Befehl dazu, dass die Seite im Inhaltsverzeichnis um 2 Zeilen länger wird



{

\renewcommand{\leftmark}{}
\renewcommand{\rightmark}{}

\vspace*{-1.5cm}
\pdfbookmark[0]{Inhalt}{pdfinhalt}
\renewcommand{\contentsname}{Inhalt}
\tableofcontents

}


\clearpage


%===================================================================
%Beginn
%===================================================================

\input{beginn/Danksagung} %danksagung optional. Wenn nicht gewünscht, dann ausgkommentieren.
\clearpage

\input{beginn/Eid} %eidesstattliche erkärung
\clearpage

\input{beginn/abkzg} %abkürzungsverzeichnis optional. Kann auch in den Anhang geschoben werden.

\cleardoublepage
\pagestyle{fancy}
\pagenumbering{arabic} %beginn mit arabischer seitenzählung

%===================================================================================
%===================================================================================


\chapter{Introduction}\label{chapter:einleitung}
% kapitel einleitung des ersten hauptteils


The problem of scheduling jobs and machines in flexible manufactoring systems has been addressed in scientific papers
for a vast amount of complex variations. For this thesis a currently relatively unexplored
set-up is considered: The scheduling of vehicles in a cyclic flexible manufactoring
system. Flexible manufactoring systems are used in a wide range of fields and in
occur in a lot of different variations. Additional machines at stages allow for a
bypass of possible bottlenecks by increasing throughput capacity. A cyclic system is
distinct from the standard version in that the depot is starting point and finish
for all jobs in the system. 

As automation is becoming increasingly relevant each day, the problem tackled here sets
the vehicles carrying jobs between the machines to be automatically guided (AGVs). The approach
considering vehicles without driver has been addressed for non cyclic problems in papers
like insertpaperhere with special consideration of problem1 and insertpaperhere focusing on 
problem2. First considerations of the cyclic variant have been made by Blazewicz and Pawlak citationSVCFMS
in which they introduced the idea of the vehicles staying in a steady cycle without any
waiting time for possible unfinished jobs and the machine schedule itself being given, allowing 
for focus on the scheduling of the AGVs alone. The first of these premises eliminates the potential of collisions between
the AGVs while the second reduces the amount of changeable variables. This thesis builds upon some of the 
findings of Blazewicz and Pawlak and implements a meta-heuristical approach by introducing a local search.

While the goal of the paper by Blazewicz and Pawlak was to solely minimize the amount of AGVs, here the approach is changed a bit:
By fixating the amount of machines in an iteration of the meta-heuristic and simply focusing on the scheduling of a given
amount of machines to minimize the makespan, a set of solutions for different amount of AGVs is received and a cost function depending
on the amount of machines and the amount of overall delay in the for a given deadline can be used to select a final
solution. If the goal remains to minimize the amount of AGVs for production without delay, the cost function can be set
to cost any constant c for each machine and Big-M per unit of delay.



%xxx optionale einleitung des ersten hauptteils


%In dieser Arbeit werden xxx behandelt. \cite{Ableitinger2013kap2} sowie \cite{Aue2013} und \cite{Beutelspacher2011} und die letzte Quelle \cite{BifieStandards}.



%\cleardoublepage
%===================================================================================
%\part{xxx Haupteil 1 xxx}\label{part:xxx-haupteil1-xxx}
%===================================================================================

%\input{hauptteil-1/hauptteil1-einleitung}

\chapter{Characteristics of the problem}\label{chapter:xxxname1xxx}
%\input{hauptteil-1/hauptteil1-kap1}
%Formulation of the problem

The problem of a cylcic flexible manufactoring system has certain limitations, which will be expanded and described in this chapter. The baseline
of the overall problem is that vehicles, that constantly go in a cycle, are used to carry jobs from one machine to the next. Extensive reasearch 
has been made in various directions considering flexible manufactoring systems. While many of these papers focus on scheduling the tasks of each of
the jobs on the machines,
this paper will only consider
a flow shop approach in which the machines are already sorted in the order, in which the jobs need to be processed. The cyclic version of
this problem sees the AGVs being routed to drive on a steady course with start and finish being the same point, therefore the AGVs
are constantly repeating laps of said track. This in turn enables us to know the position of an AGV at any given point in the system solely by
setting a starting time at which the AGV is "released". The actual position in the system is then acquired by calculating the overall time minus
the starting time modulo the lap time.

\[
Pos_{m t} = (T_{m} + t)\>\mathrm{mod}\>l
\]

Since the schedule of the jobs on the machines themselves is considered to be given, after setting the AGVs to automatically pick up the job 
which is next in the schedule on the next machine and waiting in the outgoing queue of the current machine, the only input needed to be able
to achieve a solution is a starting time for each of the AGVs.

Before it is possible to formulate an approach to solve this problem, some rules have to be set first, especially considering the handling of
jobs in outgoing queues (jobs which have been completed at a machine but have not yet been picked up by an AGV). Since the system is set to be
in constant movement, a machine cannot simply wait for a job not yet finished. Therefore a rule has to be set as to which job is to be picked out
of those waiting in line. This rule requires the machine to pick up the next job in chronological order that is both scheduled to follow the job
currently being worked on at the next stage, and waiting in the outgoing queue of the current machine e.g. if the first job is currently being
processed on a machine at the next stage, the second job would be next in line. If that job is not available for transportation, by not being in line yet
or already being transported, the third is considered and so on.

The original problem which laid the foundation for this paper, as described by AUTHOR in PAPER, differs from the problem the actually examined
problem in another feature: In the paper by AUTHOR the goal was to minimize the amount of vehicles it takes to be able to
deliver all jobs in time to satisfy a given schedule with the amount of allowed laps being given. The approach in this thesis is that there is no
given deadline known beforehand. The idea is to offer multiple solutions with differing amounts of AGVs given. Each of these solutions will assign
a makespan to a given amount of vehicles, which is defined by the sum of time units passed from the start of the problem until all jobs have been
completed and have been transported back to the depot. These solutions can then be ranked e.g. via
a cost function that takes vehicles and time needed into consideration or, if the goal is to finish ahead of a given deadline, the cost function
can be set to infinite when exceeding the threshold, to ensure that only the solution with the fewest vehicles which finishes in time will be selected.

As described in the previous paragraph, this paper reformulates the original problem posed in the paper by Blazcewicz et al to be multidimensional. This
approach has been chosen because it increases the flexibility of the whole approach. While the approach with a given deadline that must be met
is certainly not only suitable for a rare case, as these problems occur very often in all kinds of fields, a two-dimensional approach which also
looks at the total makespan introduces the possibility of a deadline that may be broken for a cost (late surcharge) or even simpler a model in
which the return is based on producing as fast as possible.









%The following example shows as to how
%picking up jobs in a different order might worsen the solution.

%Consider a problem with a single AGV and a schedule on machine n+1 set to be [1,2,3]. Jobs 2 and 3 are currently waiting to be picked up 
%in the outgoing queue of machine n, while job 1 is still currently being processed on machine n. Considering the lemma an incoming AGV has to pick
%up job 2 as it is the next job which is supposed to be processed on machine n+1 and sitting in the outgoing queue of machine n. In this case the 
%AGV now picks up job 3 instead and delivers it to machine n+1, where job 3 is now waiting in the incoming queue. While the AGV is continuing its
%lap job 1 has now been finished on machine n and is ready for pick up. As the AGV returns to machine n the job 1 is now picked up and delivered
%to the machine n+1, where it can immediately start to be processed. Somewhere during the continuation of the lap of the AGV job1 is finished on
%machine n+1 and the machine is free to process a new job. Here however it is unable to do so as in the incoming queue there is still only the
%job 3, which is supposed to be processed only after the job 2. If the AGV had picked up job 2 in the first round the job could have been started
%now, so the decision to load job 3 led to an increase in makespan that is at least the minimum of the processing time of job 2 on the second
%machine and the time it takes the AGV to pick up and deliver job 2.

%The original problem which laid the foundation for this paper, as described by AUTHOR in PAPER, differs from the problem the actually examined
%problem in another small but important feature: In the paper by AUTHOR the goal was to minimize the amount of vehicles it takes to be able to
%deliver all jobs in time to satisfy a given schedule with the amount of allowed laps being given. The approach in this thesis is that there is no
%given deadline known beforehand. The idea is to offer multiple solutions with differing amounts of AGVs given. Each of these solutions will assign
%a makespan to a given amount of vehicles, which is defined by the sum of time units passed from the start of the problem until all jobs have been
%completed and have been transported back to the depot. These solutions can then be ranked e.g. via
%a cost function that takes vehicles and time needed into consideration or, if the goal is to finish ahead of a given deadline, the cost function
%can be set to infinite when exceeding the threshold, so simply the solution with the fewest vehicles that finishes in time will be selected.

%As briefly mentioned in the previous paragraph, this paper reformulates the problem posed in the paper by AUTHOR to be multidimensional. This
%approach has been chosen because it increases the flexibility of the whole approach. While the approach with a given deadline that must be met
%is certainly not only suitable for a rare case as these problems occur very often in all kinds of fields, a two-dimensional approach which also
%looks at the total makespan introduces the possibility of a deadline that may be broken for a cost (late surcharge) or even simpler a model in
%which the return is based on producing as fast as possible.


\chapter{History of the problem in literature}\label{chapter:xxxname1_2xxx}
The category of flexible flow shop scheduling problems has a long history in the literature and has been addressed in a great amount of papers. Arthanari and Ramamurthy \cite{arth1971} are credited to be among the first to have addressed the problem in their paper "An extension of the two-machine sequencing problem". However, there were even earlier attempts at similar problems, notably in 1954 by Dantzig and Fulkerson \cite{dantzig1954}, in which they had a look at the problem of minimizing the amount of carriers to meet a fixed schedule. In the paper, they showed that the problem of finding the minimum amount of carriers needed is solveable as a linear problem.

Since then, a lot of different angles of the initial problem have been regarded; an article published by Wang \cite{wang2005} in 2005 provides a general overview of different approaches and special cases. More recently Jungwattanakit et al. \cite{jung2009} published a comparison of scheduling algorithms for flexible flow shop problems specifically looking at problems with unrelated parallel machines, setup times and dual criteria.

The biggest subcategory of flexible flow shop problems that the problem in this thesis can be subsumed under the category of cyclic flexible flow shop. Among the first to utilize this specific version of the flexible flow shop scheduling problems were Blazewicz et al. \cite{blazewicz1994} in 1994. In an unpublished paper Blazewicz and Pawlak also constructed the specific variation of the cyclic flexible flow shop that poses the foundation upon which this thesis is built \cite{blazewicz1111} and proposed a method to solve the problem using a circular arc-graph model.\cite{blazewicz198}

Since then most notably the works of Kats and Levner added insight surrounding the problem. Kats and Levner published four papers together that focused on very similar problems. Kats' paper "Minimizing the number of robots to meet a given cyclic schedule" \cite{kats1997b} shares the idea of assigning transportation vehicles to a given schedule, but the problem differs in two main points: Instead of a no-wait clause of the transportation vehicles, the jobs are not allowed to wait at the station; there is no outgoing queue allowed. The second major difference is, that although the robots have to stick to a cycle of sequence of moves, that sequence can be generated from all available moves. Therefore instead of all robots following the same cycle, each robot follows its own. This version of the cyclic flow shop is also the basis for Kats' second paper published in the same year \cite{kats1997a}, in which the researchers introduce a strongly polynomial algorithm to solve the problem.

Also in 1997, Levner, Kats and Levit proposed an improved algorithm for cyclic flow shop scheduling in a robotic cell \cite{levner1997}. The main objective in the paper is to find an optimal cycle for just a single robot. This poses an alternate approach to simply cycling the machines in the order that they are visited in the flow shop. The approach described in this paper provides a theoretical bridge between the problem focused on in this thesis with multiple vehicles on a set course and the problem described by Kats and Levner \cite{kats1997b} with multiple vehicles each of which is to be sent on an independant course.

In 2010, Levner et al published a survey about the complexity of cyclic scheduling problems in which they elaborate on the progress that has been made in the field of cyclic scheduling as a whole. They distinguish three main categories of cyclic scheduling problems: cyclic jobshop, cyclic flowshop and cyclic project scheduling problems. The problem focused upon in this thesis falls into the second category, which, according to Levner et al is NP-hard in general; however, there are specific cases, that are solveable in polynomial time \cite{levner2010}.

\chapter{Choosing an approach}\label{chapter:xxxname2xxx}
%\input{hauptteil-1/hauptteil1-kap2}
Due to the NP-completeness of the problem the general approach was to find a suitable heuristic that would be able to provide a relatively decent
solution, while maintaining a low running time. There is a wide choice of heuristical approaches described in detail in a vast amount of papers,
the idea was to find one which would be able to provide an approach useful to the problem of scheduling AGVs in the circular system. For this reason
the local search meta-heuristic was chosen as an underlying approach, in part because a neighbourhood can be relatively easily defined in a scheduling
problem and partly because a special form of the neighbourhood search, the variable neighbourhood search, opened up a new approach to the general
problem. 

The variable neighbourhood search, as closer described in PAPERHERE by AUTHOR, tries to improve the solution of the standard approach by limiting
an iteration of the search to a predefined neighbourhood and switching to a different one for the next iteration. This leads to higher coverage of
the whole solution space and a reduced chance to get stuck in a local optimum. While the problem of minimizing the amount of AGVs needed proved
to be tricky to divert into neighbourhoods to apply a variable neighbourhood search, it offered up the possibility of a reformulation of the original
problem. In the original problem the goal was solely to minimize the amount of AGVs needed to finish on time. If however the problem is changed to
have two objectives, being the minimization of the makespan and the minimization of the amount of vehicles needed, the approach of the variable
neighbourhood search can be easily applied by setting the different neighbourhoods to be all scheduling possibilities while varying the amount of
machines available. This opens up a new problem: Because a higher amount of machines will generally result in a lower makespan, the solutions gained
from different neighbourhoods are now ineligible for unreflected comparison.

Because of this problem a new approach, developed for multidimensional target functions, is introduced: A pareto frontier. AUTHOR2 wrote about the
idea of a pareto frontier extensively and showing that it offered the possibility of displaying the optima of a multidimensional problem and allowed
for the choice of a preferred solution through implementation of an additional target function (e.g. a cost function of the two dimensions
displayed) or simply by personal preference. The pareto frontier was chosen as an instrument for this thesis, because of a reason already touched
previously: Due to the nature of scheduling problems increasing the number of AGVs available to transport the jobs between the machines will in
its optimal configuration always yield a makespan that is better or as a worst case equal to the makespan received when calculating with one less
AGV. This can easily be proven by finding the optimal solution for a given amount of AGVs n and then increasing the amount. All AGVs are assigned
exactly the same starting time as the previously attained optimal solution for n AGVs. The newly added AGV can be freely allocated to any starting
time. This results in all jobs being picked up either at the exact same time as previously or, due to the newly added AGV, earlier than previously.
Therefore the total makespan will at the worst case stay the same(e.g. the newly added machine does not transport any job in the system or the
earlier transport of a job simply results in longer time in queue due to the processing time of the previous job)

In this thesis the received pareto frontier will not be filtered through the introduction of a cost function, as this simply picks one of the solutions
from the list, instead a comparison will be made for each amount of machines up to a set limit between the heuristical solution and an optimal
solution obtained through a brute-force algorithm. The difference in run time and target value will then be used to determine whether the heuristic
offers a significant reduction in the first, while maintaining only a slight reduction in the latter.

\chapter{Implementation}\label{chapter:xxxname3xxx}
%\input{hauptteil-1/hauptteil1-kap3}
Before we are able to try and approach the problem with a local search algorithm, the local search itself has to be defined more clearly. As stated
in the previous chapter, the idea is to vary the amount of machines available and find optima for all of those cases to be able to display a pareto
frontier. To be able to achieve this it is necessary to adjust the original local search, to be able to distinguish between solutions achieved with
varying amounts of AGVs, as the solutions received for the maximum makespan with a given amount of AGVs cannot be directly
compared. To circumvent this problem the local search approach is refined to iteratively increase the amount of AGVs, while not changing the amount
during an iteration. This new iterative local search consists of a set amount of "local searches" which equals the maximum amount of AGVs permitted.
In each iteration the amount of AGVs is increased by a single unit, while staying constant throughout. This approach will return a single solution for
each amount of AGVs, which is then displayed in a pareto frontier.

For the actual implementation of the model a few parameters have to be set. These include the number of iterations that the local search is going
to run for per amount of machines, the maximum amount of machines and of course all the parameters set by a specific test environment have to be
saved, namely the amount of production stages, the production machines for each stage, the amount of jobs, the lap time of an AGV, the time needed 
by an AGV to traverse between machines, the schedule of jobs on the production machines and of course the process time for each job on each machine.
As noted in a previous chapter, the only input variable is the actual starting time for each of the AGVs.

To venture a bit further into the comparability of the heuristic, this paper introduces multiple variations working on the same base construct.
The main difference is in the pertubation step which happens after each iteration that has "stranded" in a local optimum. The base version was
a simple random jump. The starting time of every AGV, except for AGVs starting at 0, is set to a random starting point and the neighbourhood
search starts anew. Before each of these pertubations the currently reached solution is compared to the best so far reached solution for the given
amount of AGVs and is saved if it is an improvement. Different from this probabilistic approach, the second variation of the heuristic set out to
be deterministic, except for the generation of the starting solution. The idea behind the differences in pertubations is to be able to achieve
comparability between efficiency and reliability. The deterministic approach sets all starting times of AGVs not starting at 0 to be starting
a set amount of units of time early(if the starting time would be negative, the time is instead counted down starting from the last possible start).
As local searches with deterministic pertubations generally suffer from the same problem, namely the pertubation either being too high, resulting
in almost random jumps, or being too low, resulting in getting stuck at a local optimum not being able to leave, the heuristic is run multiple
times with varying pertubation lengths.



\chapter{Mathematical Formulation}
In this section the paper will go into depth as to how the problem was approached from a mathematical perspective. To be able to approach the problem
with a programmed algorithm, some definitions have to be made beforehand. As described closer in previous chapters, the general idea is that AGVs
are released onto a track, where they will run in a neverending circle. Upon reaching a machine or the depot the currently loaded job is unloaded and
the job that is scheduled to be processed the earliest on the next machine is selected from the jobs waiting in the outgoing queue at the machine.
Jobs unloaded at a machine are placed in the incoming queue. Note that this queue does not have a real order as the schedule of jobs on the machines is
given, meaning that even though a job might have been the first to arrive at a machine, it might not get processed at said machine until after some
other jobs have been. The objective of the program is to minimize the maximum completion time for a given amount of AGVs \(V\). The objective function can therefore be
depicted as:

\begin{equation}
\mathrm{min}\>C_{max}
\end{equation}

For this formulation a relatively long list of constraints has to be set, which will now be explained in detail. Constraint (1) is set to make sure
that every available vehicle has to start before a full lap could have been made, since if a vehicle was to start at a later point in time it would
simply miss a full lap that could have been achieved by letting the vehicle start roundtime \(R\) units of time earlier.

\(s.t.:\)
\begin{equation}
\sum_{t=0}^{R-1}s_{v t} = 1\ldots \forall v.
\end{equation}
\begin{equation}
s_{v t}\in\{0,1\}
\end{equation}

In addition to setting the vehicle starting time to be in the range of the optimal starting time, this restriction also guarantees that each vehicle
will start at only one point in time which has to hold true to be able to determine the position of the vehicle throughout the rest of the program.


%\addtocontents{toc}{\protect\clearpage} % würde den hauptteil 2 auf eine neue seite im Inhaltsverzeichnis schieben, da seitenumbruch nach dem letzten Inhalts-Verzeichnis-Eintrag gemacht wird

%\cleardoublepage
%===================================================================================
%\part{xxx Haupteil 2 xxx}\label{part:xxx-haupteil2-xxx}
%===================================================================================


%\input{hauptteil-2/hauptteil2-einleitung}

%\chapter{Kapitel 1 des Hauptteils 2}\label{chapter:xxxname21xxx}
%\input{hauptteil-2/hauptteil2-kap1}

%\chapter{Kapitel 1 des Hauptteils 2}\label{chapter:xxxname22xxx}
%\input{hauptteil-2/hauptteil2-kap2}




%===================================================================================
%\part{...}
%===================================================================================

%hier folgenden dann die weiteren Hauptteile und Kapitel
%einfach selbst einen unterordner im hauptordner machen und die kapitel in tex-Dateien schreiben (vgl. oben)


\cleardoublepage



%===================================================================================
\part{Resümee}\label{part:resumee}
%===================================================================================

\chapter{Zusammenfassung}
%\input{hauptteil-resumee/kap-zusammenfassung}

\chapter{Ausblick}
%\input{hauptteil-resumee/kap-ausblick}


\cleardoublepage


%===================================================================================
\addpart{Literatur}
%===================================================================================

\renewcommand{\bibname}{Literatur}
\printbibliography 

\cleardoublepage

%===================================================================================
\addpart{Verzeichnisse}
%===================================================================================
\clearpage

\phantomsection %hyperverlinkung der nachfolgenden toc-zeile
\addcontentsline{toc}{chapter}{Abbildungen} %eintrag der Überschrift ins Verzeichnis

\renewcommand{\listfigurename}{Abbildungen}
\listoffigures
%alle in \begin{figure} ...  \caption{xxx} \end{figure} angeführten Objekte werden aufgeführt 
\clearpage


\phantomsection
\addcontentsline{toc}{chapter}{Tabellen}  %eintrag der Überschrift ins Verzeichnis
\renewcommand{\listtablename}{Tabellen}
\listoftables %alle in \begin{table} ...  \caption{xxx} \end{table} angeführten Objekte werden aufgeführt



\cleardoublepage

\appendix %schaltet um auf Anhang-Nummerierung A, B, ...
%===================================================================
\addpart{Anhang} %optionaler Anhang 
%===================================================================

%anhang:


\chapter{Erster Teil vom Anhang}


\chapter{Zweiter Teil vom Anhang}



{
\renewcommand*{\chapterheadstartvskip}{\vspace*{-0.5cm}}

\chapter{Oben beginnende Überschrift, falls nötig}
}




 


\end{document}