Before we are able to try and approach the problem with a local search algorithm, the local search itself has to be defined more clearly. As stated
in the previous chapter, the idea is to vary the amount of machines available and find optima for all of those cases to be able to display a pareto
frontier. To be able to achieve this it is necessary to adjust the original variable neighbourhood search, in that there is little to no gain by
constantly swapping between the neighbourhoods, as the solutions received for the maximum makespan with a given amount of AGVs cannot be directly
compared. The swapping of neighbourhoods is therefore reduced to be only once after a given amount of iterations, which is set beforehand, for a
given amount of AGVs. This in turn means that each swap of neighbourhood means that the previous neighbourhood is assigned a solution that is then
saved in the pareto frontier. As this modified approach has now drifted away considerably from the original approach of variable neighbourhood search,
the author deems it necessary to rename the heuristical approach to be more simply called "iterated neighbourhood search".

For the actual implementation of the model a few parameters have to be set. These include the number of iterations that the local search is going
to run for per amount of machines, the maximum amount of machines and of course all the parameters set by a specific test environment have to be
saved, namely the amount of production stages, the production machines for each stage, the amount of jobs, the lap time of an AGV, the time needed 
by an AGV to traverse between machines, the schedule of jobs on the production machines and of course the process time for each job on each machine.
As noted in a previous chapter, the only input variable is the actual starting time for each of the AGVs.

To venture a bit further into the comparability of the heuristic, this paper introduces multiple variations working on the same base construct.
The main difference is in the pertubation step which happens after each iteration that has "stranded" in a local optimum. The base version was
a simple random jump. The starting time of every AGV, except for AGVs starting at "0", is set to a random starting point and the neighbourhood
search starts anew. Before each of these pertubations the currently reached solution is compared to the best so far reached solution for the given
amount of AGVs and is saved if it is an improvement. Different from this probabilistic approach, the second variation of the heuristic set out to
be deterministic, except for the generation of the starting solution. The idea behind the differences in pertubations is to be able to achieve
comparability between efficiency and reliability. The deterministic approach sets all starting times of AGVs not starting at "0" to be starting
a set amount of units of time early(if the starting time would be negative, the time is instead counted down starting from the last possible start).
As local searches with deterministic pertubations generally suffer from the same problem, namely the pertubation either being too high, resulting
in almost random jumps, or being too low, resulting in getting stuck at a local optimum, not being able to leave, the heuristic is run multiple
times with varying pertubation lengths.