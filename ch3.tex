Before we are able to try and approach the problem with a local search algorithm, the local search itself has to be defined more clearly. As stated
in the previous chapter, the idea is to vary the amount of machines available and find optima for all of those cases to be able to display a pareto
frontier. To be able to achieve this it is necessary to adjust the original local search, to be able to distinguish between solutions achieved with
varying amounts of AGVs, as the solutions received for the maximum makespan with a given amount of AGVs cannot be directly
compared. To circumvent this problem the local search approach is refined to iteratively increase the amount of AGVs, while not changing the amount
during an iteration. This new iterative local search consists of a set amount of "local searches" which equals the maximum amount of AGVs permitted.
In each iteration the amount of AGVs is increased by a single unit, while staying constant throughout. This approach will return a single solution for
each amount of AGVs, which is then displayed in a pareto frontier.

For the actual implementation of the model a few parameters have to be set. These include the number of iterations that the local search is going
to run for per amount of machines, the maximum amount of machines and of course all the parameters set by a specific test environment have to be
saved, namely the amount of production stages, the production machines for each stage, the amount of jobs, the lap time of an AGV, the time needed 
by an AGV to traverse between machines, the schedule of jobs on the production machines and of course the process time for each job on each machine.
As noted in a previous chapter, the only input variable is the actual starting time for each of the AGVs.

To venture a bit further into the comparability of the heuristic, this paper introduces multiple variations working on the same base construct.
The main difference is in the perturbation step which happens after each iteration that has "stranded" in a local optimum. The base version was
a simple random jump. The starting time of every AGV, except for AGVs starting at 0, is set to a random starting point and the neighbourhood
search starts anew. Before each of these perturbations the currently reached solution is compared to the best so far reached solution for the given
amount of AGVs and is saved if it is an improvement. Different from this probabilistic approach, the second variation of the heuristic set out to
be deterministic, except for the generation of the starting solution. The idea behind the differences in perturbations is to be able to achieve
comparability between efficiency and reliability. The deterministic approach sets all starting times of AGVs not starting at 0 to be starting
a set amount of units of time early(if the starting time would be negative, the time is instead counted down starting from the last possible start).
As local searches with deterministic perturbations generally suffer from the same problem, namely the perturbation either being too high, resulting
in almost random jumps, or being too low, resulting in getting stuck at a local optimum not being able to leave, the heuristic is run multiple
times with varying perturbation lengths.

The reasoning behind the refusal to change the starting times of machines starting at zero stems from the fact that minimizing the maximum makespan
is proven to be a regular function (zitat wär gut). If all machines would start a set amount of units of time later, the objective value would simply
be the same plus the added delay and all jobs would be transported by the same vehicles and processed on the same parallel machines as before. The
regularity of the objective function requires at least one of the vehicles to start at 0, which led to the decision of simply keeping the starting
time of vehicles already starting at 0. This however does not mean that the starting time of a vehicle is set in stone once it reaches zero, it may
still change during the local search process of finding a better neighbour. If the objective value of the program is better if the vehicle in question
starts at either 1 or \(R\)-1 (the neighbours of 0), then that move will be made, if it is the best available.

The explanation of the implementation of the model in Python will be segmented into two subchapters. The first will go into extensive detail about
the program written to obtain objective values for a given set of vehicle starting times. The second subchapter focuses on the implementation of the
heuristic itself.