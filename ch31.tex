As suggested by the title of this chapter, the program written to return the objective values is a simulation of the problem. The program runs in
steps of single units of time and checks for the following events to occur in chronological order

The first events check if jobs have been completed on the machines:

\begin{itemize}
\item Is a job currently being worked on by a machine?

\item If there is, is it due to finish right now?
\end{itemize}

The second set of events looks at everything happening with the vehicles:

\begin{itemize}
\item Is a vehicle currently passing a machine/ the depot?

\item If it is passing the depot: Is the vehicle loaded?

\item Is there a job in the outgoing queue of the depot?

\item If there is, which job of the waiting jobs is the one scheduled earliest on the first machine?

\item If it is passing a machine: Is the vehicle loaded?

\item Is there a job in the outgoing queue of the machine?

\item If there is, which job of the waiting jobs is the one scheduled earliest on the next machine?
\end{itemize}

After these checks were made the final checks for the machines have to be done:

\begin{itemize}
\item Is a machine currently idle?

\item If it is, are there jobs waiting in the incoming queue of the machine?

\item Which job is the one scheduled next on this machine?
\end{itemize}
