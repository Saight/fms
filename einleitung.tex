Abstract

The problem of scheduling jobs and machines in flexible manufactoring systems has been addressed in scientific papers
for a vast amount of complex variations. One of the big issues in this category of problems is, that even very basic
variations of the problems connected with such systems are NP-complete e.g. scheduling of jobs in a flexible manufactoring
system with transportation (proven in PAPER) or scheduling of jobs on parallel machines (proven in PAPER). This thesis
tries to approach the underlying problem from an entirely different perspective: The schedule of jobs at each stage is considered
given, only the amount of transportation vehicles can be varied. In addition to the constraint of the jobs being in a set order, the
transportation vehicles are only allowed to follow a set circuit, visiting all the machines in the order in which the
jobs need to be processed, to finally return to the depot and start a new lap. This paper tries to solve the problem with
a heuristical approach derived from the idea of Variable Neighborhood Search, which is broken into its pieces to offer
an iterative neighborhood search for a two-dimensional target function. This heuristic is then applied in two ways, with
differences in the pertubation phase. In addition to the general heuristical approach, a Brute Force algorithm is used to
solve a test case to be able to compare the solutions and runtimes of the heuristics with an approach which returns the optimal
solution. The heuristic variations and the brute force algorithm were implemented using Python.

Introduction

The field of scheduling problems has been a topic of interest for researches of various fields for dozens of years already. Through
revelations of technology and economy however the field of new problems expands at a similar rate to the field of solved ones. While simple
approaches have been solved for some time now, globalisation not only leads to scheduling problems within the company, between factories 
that are distributed globally but also to having bigger factories than ever before, resulting in a higher amount of machines and more jobs
to be scheduled. With technological advancements however those bigger factories are not simply staying the same problems simply enlarged in
quantity of machines and jobs, but sophisticated new systems are being developed to be able to increase production speed, volume or reduce
the amount of manual labor needed. The idea of offering multiple machines at a single stage of the production process lead researchers to the
model of the Flexible Flow Shop. As with all flow shop problems the basic idea is that all jobs have to be processed on a set amount of machines in
the same order. As this can relatively easily lead to bottlenecks due to most jobs taking a considerable amount longer on a given machine, the idea
of parallel machines at a single stage is introduced. Each job being delivered to a stage with parallel machines may be assigned to either of
those machines to be processed, its schedule for which machines to visit becomes flexible. 

The scheduling problem focused on in this paper expands
the idea of the flexible flow shop to make use of an additional technological advancement which use is quickly spreading throughout all kinds of
industrial fields: Automation. Automation is becoming increasingly relevant as being able to replace workers with automated machines promises a
significant competitive edge. In this paper the above mentioned problem of scheduling the jobs in a flexible flow shop environment is set to be
given. The focus here lies on the problem of transportation for the jobs between the machines. With the idea of flexible flow shop including multiple
machines at each stage, the simplification of the transportation time between machines being zero becomes less and less realistic. As human resources
are costly and skilled workers would be wasted on simple transportation tasks, automation is looked towards to give a hand with this problem.
Machines such as conveyor belts or automatic guided vehicles (AGVs) are supposed to automatically transport the jobs between the machines without
any human assistance other than service and routing. 

For this thesis a currently relatively unexplored set-up is considered: A cyclic flexible manufactoring system. The cyclic system is
distinct from the standard version in that the depot is starting point and finish for all jobs in the system. This entails the possiblity of
being able to send AGVs on a steady cycle throughout the whole production area without any waiting time for possibly unfinished jobs. Waiting
for a part to carry to the next stage would be a possiblity, but it would remove one of the key features this approach offers: It is collision
free. In addition the position of each vehicle at any given time can already be set in stone solely by fixating the point in time at which the
AGV leaves the starting point for the first time. 



While the goal of the paper by Blazewicz and Pawlak was to solely minimize the amount of AGVs, here the approach is changed a bit:
By fixating the amount of machines in an iteration of the meta-heuristic and simply focusing on the scheduling of a given
amount of machines to minimize the makespan, a set of solutions for different amount of AGVs is received and a cost function depending
on the amount of machines and the amount of overall delay in the for a given deadline can be used to select a final
solution. If the goal remains to minimize the amount of AGVs for production without delay, the cost function can be set
to cost any constant c for each machine and Big-M per unit of delay.



%xxx optionale einleitung des ersten hauptteils


%In dieser Arbeit werden xxx behandelt. \cite{Ableitinger2013kap2} sowie \cite{Aue2013} und \cite{Beutelspacher2011} und die letzte Quelle \cite{BifieStandards}.

