%Vorlage für akademische Arbeiten
%erstellt von Mag. Martin Glatz
%mathematik@oehunigraz.at
%mathematik.oehunigraz.at
%ig-mathe (Studienvertretung Mathematik Uni Graz)
%Stand: März 2015

% *****************************************************************
% % ********************** Präambel *******************************
% % ***************************************************************


%===================================================================
%Dokumentklasse...
%===================================================================
\documentclass[
		fontsize=10pt,		%Standardschriftgröße auf 10 pt gesetzt
		a4paper,			%Papierformat: b=148mm h=210mm
		parskip=half-,		%kein Absatzeinzug, stattdessen Höhe einer halben Zeile.
		twoside,			%zweiseitiges Layout (Seitenzahlen außen)
		headings=big,	%große überschriften samt abstände
		appendixprefix=true,
		]{scrreprt}		%koma-script: europäische Formatvorlage für  scrreprt (chapter-ebene vorhanden)



%===================================================================
%Standard-Codierungen und Schriften...
%===================================================================

\usepackage[utf8]{inputenc}	%Zeichencodierung in den Textfiles
\usepackage[english,ngerman]{babel}	%deutsche Rechtschreibung, Worttrennung
\usepackage[T1]{fontenc} 		%Schriftcodierung
\usepackage{lmodern}			%LaTeX-Schriftpaket laden (lmodern)
%\usepackage{microtype} % für zeilenumbrüche etc, verträgt sich nicht mit  allen schriften. Einfach probieren oder in der dokumentation nachlesen



\input{Vorspann/pakete}



\setcounter{secnumdepth}{4}
\setcounter{tocdepth}{4}

\usepackage[
	colorinlistoftodos,
	textwidth=3.5cm,
	textsize=footnotesize]{todonotes}	%notizen im pdf



%===================================================================
%Einbinden der anderen Präambel-Dateien
%===================================================================



\input{Vorspann/design}
\input{Vorspann/befehle-tabellen}
\input{Vorspann/befehle-mathe}
\input{Vorspann/befehle-umgebungen}


%===================================================================
%Setup für Verlinkungen und Farben etc.
%===================================================================
\usepackage[
	linktocpage=true, 
	colorlinks=true,
	linkcolor=farbe1,
	citecolor=farbe1,
	urlcolor=farbe1
	]{hyperref}

%===================================================================
%Einstellungen Literatur: Momentan Numerisch. Bei Bedarf ändern
%===================================================================

\usepackage
	[
	backend=biber,	%		sortierprogramm
	%backend=bibtex, %alternatives sortierprogramm. hilfsdaten vor umstellung löschen
	style=numeric-comp,
	maxbibnames=99,
	maxcitenames=1,
	urldate=long,
	sorting=none
	]
	{biblatex}
\addbibresource{literatur-da.bib} %laden der literaturliste
\DefineBibliographyStrings{ngerman}{andothers={et\addabbrvspace al\adddot}}
\DefineBibliographyStrings{ngerman}{urlseen={abgerufen am}}









% *****************************************************************
% % ********************** Präambel *******************************
% % ***************************************************************

\begin{document}

\pagenumbering{Roman}

\input{beginn/Titelseite} %titelblatt
\clearpage

\input{beginn/Abstract} %abstract englisch
\clearpage

\input{beginn/Kurzfassung} %abstract deutsch
\clearpage


%===================================================================
%Gestaltung Inhaltsverzeichnis
%===================================================================


\setcounter{secnumdepth}{4}	%nummerierte ebenen
\setcounter{tocdepth}{4}	%angeführte ebenen im verzeichnis

\tocsetup{
	title/font = \huge\color{farbe1},
	part/font = \large\bfseries\color{farbe1},
	part/number/width=8mm,
	part/number/after = .,
	chapter/number/after = .,
	chapter/indent=8mm,
	chapter/number/width=8mm,	
	section/number/after = .,
	section/indent = 16mm,
	section/number/width=10mm,
	subsection/indent=26mm,
	subsection/number/after = .,
	subsection/number/width=10mm,
	subsubsection/indent=36mm,
	subsubsection/number/after = .,
	subsubsection/number/width=13mm
	}
\listofsetup{lot}{
	parskip=0cm,
	title/font = \huge\color{farbe1},
	table/number/after = .,
	table/number/width=13mm,	
	}	
\listofsetup{lof}{
	parskip=0cm,
	title/font = \huge\color{farbe1},
	figure/number/after = .,	
	figure/number/width=13mm,}



%Nützliche Befehle fürs toc: 

%\addtocontents{toc}{\protect\clearpage} % fügt Zeilenumbruch im Inhaltsverzeichnis nach dem vorangegangenen Eintrag durch. Muss also im Fließtext bzw. zwischen den Überschriften stehen

%\addtocontents{toc}{\protect\enlargethispage{2\baselineskip}} %fügt nach dem vorigen toc-Eintrag den Befehl dazu, dass die Seite im Inhaltsverzeichnis um 2 Zeilen länger wird



{

\renewcommand{\leftmark}{}
\renewcommand{\rightmark}{}

\vspace*{-1.5cm}
\pdfbookmark[0]{Inhalt}{pdfinhalt}
\renewcommand{\contentsname}{Inhalt}
\tableofcontents

}


\clearpage


%===================================================================
%Beginn
%===================================================================

\input{beginn/Danksagung} %danksagung optional. Wenn nicht gewünscht, dann ausgkommentieren.
\clearpage

\input{beginn/Eid} %eidesstattliche erkärung
\clearpage

\input{beginn/abkzg} %abkürzungsverzeichnis optional. Kann auch in den Anhang geschoben werden.

\cleardoublepage
\pagestyle{fancy}
\pagenumbering{arabic} %beginn mit arabischer seitenzählung

%===================================================================================
%===================================================================================


\chapter{Einleitung}\label{chapter:einleitung}
% kapitel einleitung des ersten hauptteils


The problem of scheduling jobs and machines in flexible manufactoring systems has been addressed in scientific papers
for a vast amount of complex variations. For this thesis a currently relatively unexplored
set-up is considered: The scheduling of vehicles in a cyclic flexible manufactoring
system. Flexible manufactoring systems are used in a wide range of fields and in
occur in a lot of different variations. Additional machines at stages allow for a
bypass of possible bottlenecks by increasing throughput capacity. A cyclic system is
distinct from the standard version in that the depot is starting point and finish
for all jobs in the system. 

As automation is becoming increasingly relevant each day, the problem tackled here sets
the vehicles carrying jobs between the machines to be automatically guided (AGVs). The approach
considering vehicles without driver has been addressed for non cyclic problems in papers
like insertpaperhere with special consideration of problem1 and insertpaperhere focusing on 
problem2. First considerations of the cyclic variant have been made by Blazewicz and Pawlak citationSVCFMS
in which they introduced the idea of the vehicles staying in a steady cycle without any
waiting time for possible unfinished jobs and the machine schedule itself being given, allowing 
for focus on the scheduling of the AGVs alone. The first of these premises eliminates the potential of collisions between
the AGVs while the second reduces the amount of changeable variables. This thesis builds upon some of the 
findings of Blazewicz and Pawlak and implements a meta-heuristical approach by introducing a local search.

While the goal of the paper by Blazewicz and Pawlak was to solely minimize the amount of AGVs, here the approach is changed a bit:
By fixating the amount of machines in an iteration of the meta-heuristic and simply focusing on the scheduling of a given
amount of machines to minimize the makespan, a set of solutions for different amount of AGVs is received and a cost function depending
on the amount of machines and the amount of overall delay in the for a given deadline can be used to select a final
solution. If the goal remains to minimize the amount of AGVs for production without delay, the cost function can be set
to cost any constant c for each machine and Big-M per unit of delay.



%xxx optionale einleitung des ersten hauptteils


%In dieser Arbeit werden xxx behandelt. \cite{Ableitinger2013kap2} sowie \cite{Aue2013} und \cite{Beutelspacher2011} und die letzte Quelle \cite{BifieStandards}.




\cleardoublepage
%===================================================================================
\part{xxx Haupteil 1 xxx}\label{part:xxx-haupteil1-xxx}
%===================================================================================

%\input{hauptteil-1/hauptteil1-einleitung}

\chapter{Kapitel 1 des Hauptteils 1}\label{chapter:xxxname1xxx}
%\input{hauptteil-1/hauptteil1-kap1}

\chapter{Kapitel 1 des Hauptteils 2}\label{chapter:xxxname2xxx}
%\input{hauptteil-1/hauptteil1-kap2}

\chapter{Kapitel 1 des Hauptteils 3}\label{chapter:xxxname3xxx}
%\input{hauptteil-1/hauptteil1-kap3}



%\addtocontents{toc}{\protect\clearpage} % würde den hauptteil 2 auf eine neue seite im Inhaltsverzeichnis schieben, da seitenumbruch nach dem letzten Inhalts-Verzeichnis-Eintrag gemacht wird

\cleardoublepage
%===================================================================================
\part{xxx Haupteil 2 xxx}\label{part:xxx-haupteil2-xxx}
%===================================================================================


%\input{hauptteil-2/hauptteil2-einleitung}

\chapter{Kapitel 1 des Hauptteils 2}\label{chapter:xxxname21xxx}
%\input{hauptteil-2/hauptteil2-kap1}

\chapter{Kapitel 1 des Hauptteils 2}\label{chapter:xxxname22xxx}
%\input{hauptteil-2/hauptteil2-kap2}




%===================================================================================
%\part{...}
%===================================================================================

%hier folgenden dann die weiteren Hauptteile und Kapitel
%einfach selbst einen unterordner im hauptordner machen und die kapitel in tex-Dateien schreiben (vgl. oben)


\cleardoublepage



%===================================================================================
\part{Resümee}\label{part:resumee}
%===================================================================================

\chapter{Zusammenfassung}
%\input{hauptteil-resumee/kap-zusammenfassung}

\chapter{Ausblick}
%\input{hauptteil-resumee/kap-ausblick}


\cleardoublepage


%===================================================================================
\addpart{Literatur}
%===================================================================================

\renewcommand{\bibname}{Literatur}
\printbibliography 

\cleardoublepage

%===================================================================================
\addpart{Verzeichnisse}
%===================================================================================
\clearpage

\phantomsection %hyperverlinkung der nachfolgenden toc-zeile
\addcontentsline{toc}{chapter}{Abbildungen} %eintrag der Überschrift ins Verzeichnis

\renewcommand{\listfigurename}{Abbildungen}
\listoffigures
%alle in \begin{figure} ...  \caption{xxx} \end{figure} angeführten Objekte werden aufgeführt 
\clearpage


\phantomsection
\addcontentsline{toc}{chapter}{Tabellen}  %eintrag der Überschrift ins Verzeichnis
\renewcommand{\listtablename}{Tabellen}
\listoftables %alle in \begin{table} ...  \caption{xxx} \end{table} angeführten Objekte werden aufgeführt



\cleardoublepage

\appendix %schaltet um auf Anhang-Nummerierung A, B, ...
%===================================================================
\addpart{Anhang} %optionaler Anhang 
%===================================================================

%anhang:


\chapter{Erster Teil vom Anhang}


\chapter{Zweiter Teil vom Anhang}



{
\renewcommand*{\chapterheadstartvskip}{\vspace*{-0.5cm}}

\chapter{Oben beginnende Überschrift, falls nötig}
}




 


\end{document}